\pdfoutput=1
\documentclass[11pt]{article}

% Page and typography
\usepackage[margin=1in]{geometry}
\usepackage[T1]{fontenc}
\usepackage[utf8]{inputenc}
\usepackage{lmodern}
\usepackage{microtype}
\usepackage{hyperref}

% Math and refs
\usepackage{amsmath,amssymb,mathtools,amsfonts}
\usepackage{amsthm}
\usepackage{bm}

% Theorem-like environments
\theoremstyle{definition}
\newtheorem{definition}{Definition}
\newtheorem{proposition}{Proposition}
\newtheorem{example}{Example}
\theoremstyle{remark}
\newtheorem*{remark*}{Remark}

% Macros
\newcommand{\R}{\mathbb{R}}
\newcommand{\sphere}{S^{N-1}}
\newcommand{\uh}{\hat{u}}
\newcommand{\pv}{\vec{p}}
\newcommand{\xv}{\vec{x}}
\newcommand{\vv}{\vec{v}}
\newcommand{\vh}{\hat{v}}
\newcommand{\dd}{\,\mathrm{d}}
\newcommand{\gv}{\vec{g}}
\newcommand{\gray}{\vec{g}_{\mathrm{ray}}}

\setlength{\parskip}{1em}
\setlength{\parindent}{0pt}

\title{Ray-Based Modeling of Gravitational Fields}
\author{Gerard Torrent--Gironella\\ \small\texttt{gerard@generacio.com}}
\date{November 8, 2025}

\begin{document}
\maketitle

\begin{abstract}
A novel framework for modeling gravitational fields is developed using a 
ray-based approach. At any observation point, the field is defined as the 
direction-weighted aggregation of line-integrated mass contributions along all 
incoming directions. In the static case, the formulation reproduces the 
classical Newtonian gravitational field and shows potential to replicate also 
relativistic scenarios. The approach represents an initial step in ray-based 
gravitational modeling, open to further refinement and generalization.
\end{abstract}

\section{Introduction}

This paper presents a preliminary exploration of ray-based representation of 
the gravitational field. The viewpoint is strictly local at the observation 
point $\pv$, which is assumed to access only the mass traversed by each 
incoming direction.
With no other knowledge about global distribution, the gravitational 
field at $\pv$ is defined to be the aggregation of these directional 
contributions multiplied by an absorption factor. For clarity, the amount of 
mass traversed by a ray is obtained as the line integral of the mass density 
over that ray. As a heuristic---and only as a possible explanatory device, not 
an ontological claim---one may picture non-interacting \emph{tracers} of 
invariant speed $c$ that traverse matter without perturbing it and carry a 
counter proportional to the traversed mass.

The static case is first formulated by defining the field as a direction-weighted 
integral of line-integrated density. The definition is illustrated with a simple 
example, for which the field reduces to the familiar inverse-square law. We then 
establish the equivalence between the ray-based expression and the classical 
Newtonian formulation.

Next, we incorporate finite propagation and uniform source motion. The kinematics 
is organized through special-relativistic aberration of directions between the 
observation frame and a co-moving integration frame, together with retarded time 
along each ray. This leads to a mixed-frame definition that can be reformulated 
entirely within the moving frame. A worked example ilustrates the relativistic 
case using both the original definition and the alternative formulation.

We conclude with a brief final section that merely points to possible avenues 
for further investigation. These pointers are deliberately high-level and 
speculative, offered as indications of where the framework might be extended 
rather than as claims or detailed developments.

\section{Static case}

\begin{definition}

Let $\rho(\xv)$ be a mass density supported in a domain $V\subset\R^N$, $N\ge 3$. 
Outside this domain the density is zero, $\rho(\xv)=0$ for $\xv\notin V$.

We define the \emph{ray-based gravitational field} by
\begin{equation}
\boxed{
\gray(\pv) = 
-G \int_{\Omega} \left(\int_{L(\pv,\uh)} \rho(\xv)\,\dd \ell\right)\uh\,\dd\Omega}
\label{eq:rayfield-static}
\end{equation}

where $\pv \in \R^N$, $\Omega=\sphere$ is the unit angular domain, $L(\pv,\uh)$ 
is the ray $\{\,\pv+c\;t\,\uh:\ t\le 0\,\}$, $\dd\ell$ is the arc length along 
the ray, and $G$ is a coupling factor.

\end{definition}
\vspace{1em}

\begin{example}

Let $V$ be a solid sphere of radius $R$ and total mass $M$ with constant density 
$\rho_0=\frac{3M}{4\pi R^3}$. Compute the ray-based gravitational field at a 
point located at distance $D > R$ from the sphere's center.

We perform a translation to place the point $\pv$ at the origin, followed by a 
rotation to place the sphere's center along the positive z-axis. These changes 
result in $\pv=\vec{0}$, and the center of the sphere at $\vec{c}=D\,\hat{d}$, 
where $\hat{d}=\hat{z}$ is the unit vector from $\pv$ to the sphere's center.

For each unit direction $\uh$ with polar angle $\theta$ measured 
from $\hat{d}$, parameterize the ray by arc length $s = -c\;t$ with $s\ge 0$:
\[
\xv(\uh,s)=\pv-s\,\uh=-s\,\uh,
\qquad
\dd\ell=\dd s
\]

The ray intersects the sphere when $\|\xv(\uh,s)-\vec{c}\|=R$, i.e.
\[
\|-s\,\uh-D\,\hat{d}\|^2 = s^2 + 2 D s\,(\uh\!\cdot\!\hat{d}) + D^2 = R^2
\]

Writing $\cos\theta=\uh\!\cdot\!\hat{d}$, the intersection parameters are the roots
\[
s_\pm(\theta)=-D\cos\theta \pm \sqrt{R^2 - D^2\sin^2\theta}
\]

which are real iff $D\sin\theta\le R$, i.e. $\theta\in[\theta_0,\pi]$ with 
$\theta_0=\pi - \arcsin(R/D)$. For such directions, the line-integrated density 
equals the chord length inside the sphere times $\rho_0$:
\[
\int_{L(\pv,\uh)}\rho_0\,\dd\ell
=\int_{s_-}^{s_+}\rho_0\,\dd s
=\rho_0\,L(\theta),
\qquad
L(\theta)=s_+ - s_- = 2\sqrt{R^2 - D^2\sin^2\theta}
\]

By symmetry, $\gray(\pv)$ is parallel to $\hat{d}$. Using spherical coordinates 
around $\hat{d}$, $\dd\Omega=\sin\theta\,\dd\theta\,\dd\phi$, the ray-based 
field is
\[
\gray(\pv)
= -G\,\rho_0\,\hat{d}\int_0^{2\pi}\!\!\int_{\theta_0}^{\pi}
L(\theta)\,\underbrace{\cos\theta}_{\uh\cdot\hat{d}}\,\underbrace{\sin\theta\,\dd\theta\,\dd\phi}_{\dd\Omega}
\]

Substituting $L(\theta)=2\sqrt{R^2 - D^2\sin^2\theta}$,
\[
\gray(\pv)
= -2G\,\rho_0\,\hat{d}\int_0^{2\pi}\!\!\int_{\theta_0}^{\pi}
\sqrt{R^2 - D^2\sin^2\theta}\,\cos\theta\,\sin\theta\,\dd\theta\,\dd\phi
\]

Evaluate the $\phi$-integral and set
\[
I:=\int_{\theta_0}^{\pi}
\sqrt{R^2 - D^2\sin^2\theta}\,\cos\theta\,\sin\theta\,\dd\theta
\]

With $u=R^2 - D^2\sin^2\theta$ we have $\dd u=-2D^2\sin\theta\cos\theta\,\dd\theta$ and
\[
I=\frac{-1}{2D^2}\int_{u(\theta_0)}^{u(\pi)} u^{1/2}\,\dd u
=\frac{-1}{2D^2}\cdot\frac{2}{3}\Big[u^{3/2}\Big]_{u=0}^{u=R^2}
=-\frac{R^3}{3D^2}
\]

Hence
\[
\gray(\pv)=-2G\,\rho_0\,\hat{d}\,2\pi\,I
= \frac{4\pi G\rho_0 R^3}{3D^2}\,\hat{d}
= \,G\,\frac{M}{D^2}\,\hat{d}
\]

since $\rho_0=3 M/(4\pi R^3)$. Therefore, the ray-based 
field reproduces the classical inverse-square law outside the sphere:
\[
\gv(\pv)=G\,\dfrac{M}{D^2}\,\hat{d}
\]

\end{example}
\vspace{1em}

\begin{proposition}[Equivalence]\label{prop:equivalence}

Let $N\ge 3$ and let $\rho\in L^1(\R^N)$ have compact support $V$. 

For any $\pv\notin \operatorname{supp}\rho$, the ray-based field 
\eqref{eq:rayfield-static},
\[
\gray(\pv)=
-G\int_{\Omega}\!\left(\int_{L(\pv,\uh)} \rho(\xv)\,\dd\ell\right)\uh\,\dd\Omega
\]

coincides with the Newtonian field \cite{LandauLifshitz1},
\begin{equation}
\gv(\pv)=-G\int_V \frac{\pv-\xv}{\|\pv-\xv\|^N}\,\rho(\xv)\,\dd V
\label{eq:newtonian}
\end{equation}

\end{proposition}

\begin{proof}

To demonstrate the equivalence, we will expand the line integral in the ray-based 
field definition, then we will rewrite the classical Newtonian field in spherical 
coordinates, and finally we will compare the two expressions to show they are 
identical.

\emph{Step 1} -- Expand the line integral in the ray-based field.

In spherical coordinates centered at $\pv$, any point $\xv$ over the ray is 
parameterized by the arc length by:
\[
\xv = \pv - s \uh
\]

where $s = \|\pv - \xv\| \geq 0$ is the radial distance, and $\uh$ is the unit 
vector indicating the direction from $\xv$ to $\pv$.

The differential arc length along the ray $L(\pv, \uh)$ is $\dd\ell = \dd s$. 
Substituting this into the ray-based field:
\[
\gray(\pv) = -G \int_{\Omega} \left( \int_{I(\uh)} \rho(\pv - s \uh) \, \dd s \right) \uh \, \dd\Omega
\]

where $I(\uh)$ is the interval of $s$ such that $\pv - s \uh \in V$.

\emph{Step 2} -- Express the Newtonian field in spherical coordinates.

In the Newtonian field, substitute $\xv = \pv - s \uh$ and $\|\pv - \xv\| = s$. 
Then:
\[
\frac{\pv - \xv}{\|\pv - \xv\|^N} = \frac{s \uh}{|s|^N} = \frac{\uh}{|s|^{N-1}}
\]

The volume element in spherical coordinates is $\dd V = |s|^{N-1} \dd s \, \dd\Omega$. 
Substituting these into the Newtonian field:
\[
\gv(\pv) = -G \int_V \frac{\uh}{|s|^{N-1}} \, \rho(\pv - s \uh) \, |s|^{N-1} \dd s \, \dd\Omega
\]

Simplifying:
\[
\gv(\pv) = -G \int_{\Omega} \int_{I(\uh)} \uh \, \rho(\pv - s \uh) \, \dd s \, \dd\Omega
\]

\emph{Step 3} -- Verify equivalence.

The ray-based field is:
\[
\gray(\pv) = -G \int_{\Omega} \left( \int_{I(\uh)} \rho(\pv - s \uh) \, \dd s \right) \uh \, \dd\Omega
\]

The Newtonian field is:
\[
\gv(\pv) = -G \int_{\Omega} \int_{I(\uh)} \uh \, \rho(\pv - s \uh) \, \dd s \, \dd\Omega
\]

By the Tonelli and Fubini theorems (\cite{Rudin}, \cite{Folland}), the order of 
integration can be interchanged under the given assumptions 
(e.g., $\rho \in L^1(\R^N)$ with compact support or sufficient decay at infinity). 
Thus, the two expressions are identical.

\end{proof}

\section{Relativistic case}

Building upon the static case, this section explores the relativistic extension 
of the ray-based gravitational field.
In contrast to the static case, where closed-form solutions were derived and 
shown to be equivalent to the classical Newtonian gravitational field, the 
relativistic case does not yet yield results of this kind. Instead, we present 
preliminary indications and formulations that suggest the framework may be 
extended to more general scenarios.

We consider two reference frames:

\begin{itemize}
\item $S$ - \emph{Rest frame}: The reference frame in which the gravitational 
    field is measured. The evaluation point $\pv$ is at rest and receives 
    isotropic radiation of rays with constant velocity $c$.
\item $S'$ - \emph{Moving frame}: The reference frame in which the mass moves 
    at velocity $\vv=v\,\vh$ relative to $S$.
\end{itemize}

Rays are parameterized based on their direction $\uh$ and time $t$ (negative 
values represent the past, and positive values represent the future).

In the $S$ frame, the ray is parameterized as
$$
r(\uh,t)=\pv+t\,c\,\uh, \quad t\le 0
$$

In the $S'$ frame, the same ray is parameterized as
$$
r'(\uh,t)=\pv+t\,c\,h(\uh;\vv), \quad t\le 0
$$

where $h()$ is the angular aberration coefficient.
When the mass moves with velocity $\vv$ relative to the 
observation point, the angle of the ray observed at $\pv$ differs from the 
angle of the ray that has traversed the mass. This discrepancy is called angular 
aberration and is modeled using the Lorentz transformation.

\subsection{Angular aberration}

The results presented in this section are well-established and can be found in 
\cite{LandauLifshitz2}. These results will serve as a foundational basis for the 
relativistic formulations and derivations developed in subsequent sections.

Given a velocity $\vv = v\,\vh$, we denote the decomposition of any 
vector $\vec{w}$ into its projections along $\vv$ and its perpendicular plane as
$$
\vec{w} = \vec{w}_\perp + \vec{w}_\parallel = |\vec{w}|\sin(\theta)\;\vh_\perp + |\vec{w}|\cos\theta\;\vh, \quad \cos\theta=\hat{w}\cdot\vh
$$

The function $h()$ transforms a direction $\uh$ in $S$ (the rest frame) to 
the corresponding direction in $S'$ (the moving frame).
$$
h(\uh;\vv) =
\frac{\sqrt{1-\beta^2}\;\uh_\perp+(\cos\theta-\beta)\,\vh}{1-\beta\cos\theta},
\qquad \beta=\frac{v}{c},
\quad \cos\theta=\uh\cdot\vh
$$

The inverse transformation ($S'\!\to\!S$) is
$$
h^{-1}(\uh';\vv)=
\frac{\sqrt{1-\beta^2}\;\uh'_\perp+(\cos\theta'+\beta)\,\vh}{1+\beta\cos\theta'},
\qquad \beta=\frac{v}{c},
\quad \cos\theta'=\uh'\cdot\vh
$$

If the polar axis is aligned with $\vv$, i.e., $\vh\parallel\hat{z}$, then
$$
\phi'=\phi,\qquad
\cos\theta'=\frac{\cos\theta-\beta}{1-\beta\cos\theta},\qquad
\sin\theta'=\frac{\sqrt{1-\beta^{2}}\;\sin\theta}{1-\beta\cos\theta}
$$

$$
\phi=\phi',\qquad
\cos\theta=\frac{\cos\theta'+\beta}{1+\beta\cos\theta'},
\qquad
\sin\theta=\frac{\sqrt{1-\beta^2}\sin\theta'}{1+\beta\cos\theta'}
$$

The Jacobian of the solid angle transformation is
$$
\frac{d\Omega'}{d\Omega}=\frac{1-\beta^2}{(1-\beta\cos\theta)^2}
\quad\Longrightarrow\quad
\frac{d\Omega}{d\Omega'}=\frac{(1+\beta\cos\theta')^2}{1-\beta^2}
$$

\subsection{Ray-based gravitational field}

\begin{definition}

Let $\rho(\xv)$ be a mass density supported in a domain $V\subset\R^N$, $N\ge 3$. 
Outside this domain the density is zero, $\rho(\xv)=0$ for $\xv\notin V$.

We define the \emph{ray-based gravitational field} at observation time $t_0$ by
\begin{equation}
\boxed{
\gray(\pv,t_0)=
-G\int_{\Omega}\!\left(\int_{L(\pv,\,h(\uh;\vv))}\!\rho(\xv',t_{\mathrm{ret}})\,\dd\ell'\right)\uh\,\dd\Omega,
\quad t_{\mathrm{ret}}=t_0-\ell'/c}
\label{eq:rayfield-rel}
\end{equation}

where $\pv \in \R^N$, $\Omega=\sphere$ is the unit angular domain, $L(\pv,\uh)$ 
is the ray $\{\,\pv+c\;t\,\uh:\ t\le 0\,\}$, $\dd\ell$ is the arc length along 
the ray, $G$ is a coupling factor, and $h(\uh; \vv)$ is the angular aberration 
function that transforms the direction $\uh$ to the corresponding direction in 
the moving frame.

We observe that the calculation of the ray-based gravitational field is 
performed in two reference frames. The contribution of each ray is computed 
in the frame $S'$, and the summation of contributions is carried out in the 
frame $S$. The following result reformulates the definition entirely in $S'$.

\end{definition}
\vspace{1em}

\begin{proposition}[Alternative formulation]

Let $\rho(\xv)$ be a mass density supported in a domain $V\subset\R^N$, $N\ge 3$. 
Outside this domain, the density is zero, $\rho(\xv)=0$ for $\xv\notin V$.

The ray-based gravitational field at observation time $t_0$ is:
\begin{equation}
\boxed{
\gray(\pv,t_0)=\frac{-G}{1-\beta^2}\int_{\Omega'}\!\left(\int_{L(\pv,\uh')}\rho(\xv',t_{\mathrm{ret}})\,\dd\ell'\right)
\left(\sqrt{1-\beta^2}\,\uh'_\perp+(\cos\theta'+\beta)\,\vh\right)(1+\beta\cos\theta')\,\dd\Omega'}
\label{eq:rayfield-rel-alt}
\end{equation}
\end{proposition}

\begin{proof}

We apply a change of variables to the ray-based gravitational field definition 
to obtain the proposition statement.

Starting from the original definition \eqref{eq:rayfield-rel}:
$$
\gray(\pv,t_0)
= -G \int_{\Omega} \left( \int_{L(\pv,\,h(\uh;\vv))} \rho(\xv',t_{\text{ret}})\, d\ell' \right) \uh\; \dd\Omega
$$

we perform the change of variables $\uh'=h(\uh; \vv)$, 
i.e., $\uh = h^{-1}(\uh';\vv)$, and use the Jacobian 
$d\Omega = \frac{d\Omega}{d\Omega'}\, d\Omega'$ with
$$
h^{-1}(\uh';\vv)=
\frac{\sqrt{1-\beta^2}\;\uh'_\perp+(\cos\theta'+\beta)\,\vh}{1+\beta\cos\theta'},
\qquad
\frac{d\Omega}{d\Omega'}=\frac{(1+\beta\cos\theta')^2}{1-\beta^2}.
$$

Thus, we have:
$$
\gray(\pv,t_0)
= -G \int_{\Omega'}\!\left(\int_{L(\pv,\,\uh')} \rho(\xv',t_{\text{ret}})\, \dd\ell' \right)\,
h^{-1}(\uh';\vv)\;
\frac{d\Omega}{d\Omega'}\; \dd\Omega'
$$

and, simplifying the factors we obtain the final expression:
$$
\gray(\pv,t_0)
= \frac{-G}{1-\beta^2} \int_{\Omega'}\!\left(\int_{L(\pv,\,\uh')} \rho(\xv',t_{\text{ret}})\, \dd\ell' \right)
\left(\sqrt{1-\beta^2}\;\uh'_\perp+(\cos\theta'+\beta)\,\vh\right)
(1+\beta\cos\theta')\; \dd\Omega'
$$

\end{proof}

\subsection{Examples}

We calculate the ray-based gravitational field at time $t_0$ generated by a 
sphere of radius $R$, mass $M$, and uniform density moving at a constant 
velocity $\vv$, acting on a point $\pv$ located at a distance $D$ from the 
center of the sphere, with $D > R$. The sphere is moving towards the 
observation point.

We perform a translation to place the point $\pv$ at the origin, followed 
by a rotation to align the $z$-axis with $\vv$. These transformations 
result in:
\begin{itemize}
\item The point $\pv$ remains at the origin, $\pv = (0,0,0)$,
\item The velocity vector of the sphere is $\vv = -v\,\hat{z}$, with $v \ge 0$
\item At $t_0$ the center of the sphere is located at $(0, 0, D) = \vec{d} = D \hat{d} = D \hat{z}$.
\end{itemize}
\vspace{1em}

\begin{example}[mixed-frames]\label{example:mixed_frames}
We resolve the case using the ray-based field definition \eqref{eq:rayfield-rel}.

We parametrize the ray using the aberrated direction $h(\uh;\vv)$ while
keeping the sphere worldline as in $S$ (hence “mixed-frames”). We seek the
intersection points between the sphere and the ray.
\[
\left.
\begin{array}{rrl}
\text{Ray:} & \xv &= \pv+t\,c\,h(\uh,\vv)\\[3pt]
\text{Sphere:} & x^2+y^2+(z - D + v\, t)^2 &= R^2
\end{array}
\right\},
\quad t \le 0
\]

We change coordinates using $\ell = -t\,c$ (arc length):
\[
\left.
\begin{array}{rrl}
\text{Ray:} & \xv &= \pv-\ell\,h(\uh,\vv)\\[3pt]
\text{Sphere:} & x^2+y^2+(z-D-\beta\, \ell)^2 &= R^2
\end{array}
\right\},
\quad \ell \ge 0
\]

Aberration formulas apply for $\vv$ aligned to the polar axis, but this is not 
the current case, $\vv=-v\,\hat{z}$. However, we can handle this case by
replacing $\beta\to-\beta$ everywhere in the aberration formulas. Thus,
$$
  h(\uh;-v \hat{z})=
  \dfrac{\sqrt{1-\beta^2}\,\uh_\perp+(\cos\theta+\beta)\,\hat{z}}{1+\beta\cos\theta}.
$$
$$
  \cos\theta'=\dfrac{\cos\theta+\beta}{1+\beta\cos\theta},\quad
  \sin\theta'=\dfrac{\sqrt{1-\beta^{2}}\,\sin\theta}{1+\beta\cos\theta}
$$
$$
  \dfrac{d\Omega'}{d\Omega}=\dfrac{1-\beta^2}{(1+\beta\cos\theta)^2}
  \quad\Longrightarrow\quad
  \frac{d\Omega}{d\Omega'}=\frac{(1-\beta\cos\theta')^2}{1-\beta^2}
$$

We write $\uh$ in angular coordinates,
\[
\uh = \big(\sin\theta\cos\phi,\ \sin\theta\sin\phi,\ \cos\theta\big)
\quad \theta \in [0,\pi], \ \phi \in [0,2\pi)
\]

We expand $\xv$ in these coordinates, using the aberration formula for 
$\vv=-v\,\hat{z}$ (see remark above):
\[
\xv(\uh, \ell) = \Big(-\ell\,\frac{\sqrt{1-\beta^2}}{1+\beta\cos\theta}\sin\theta\cos\phi, \ -\ell\,\frac{\sqrt{1-\beta^2}}{1+\beta\cos\theta}\sin\theta\sin\phi,\ -\ell\,\frac{\cos\theta+\beta}{1+\beta\cos\theta}\Big)
\]

Finally, we substitute $\xv(\uh,\ell)$ into the sphere's equation. 
This yields a quadratic polynomial in $\ell$.
\[
\ell^2 \frac{1-\beta^2}{(1+\beta\cos\theta)^2}\sin^2\theta+\left(-\frac{\cos\theta+\beta}{1+\beta\cos\theta}\ell-\beta\ell-D\right)^2=R^2
\]

Reorganizing terms by the coefficient of $\ell$, we obtain:
\[
A(\theta)\,\ell^2+2B(\theta)\,\ell+C(\theta)=0,
\quad
\begin{cases}
A(\theta)=(1-\beta^2)(1-\cos^2\theta) + (\cos\theta + \beta^2\cos\theta + 2\beta)^2\\[2pt]
B(\theta)=D\,(1+\beta\cos\theta)\,(\cos\theta + \beta^2\cos\theta + 2\beta)\\[2pt]
C(\theta)=(1+\beta\cos\theta)^2(D^2-R^2)
\end{cases}
\]

The chord length traversed by the ray is:
\[
L(\theta) = 2\sqrt{\left(\frac{B(\theta)}{A(\theta)}\right)^2 - \frac{C(\theta)}{A(\theta)}}
\]

The range of $\theta$ is determined by ensuring the discriminant is positive. 
Let $\theta_0$ be the first root less than $\pi$ such that the discriminant 
is zero. 

The gravitational field is:
\[
\gray(\pv, 0) = -G \rho_{\scriptscriptstyle 0} \int_0^{2\pi}\int_{\theta_0}^{\pi} L(\theta) \left(\sin\theta\cos\phi\,\hat{\mathbf{i}} + \sin\theta\sin\phi\,\hat{\mathbf{j}}+ \cos\theta\,\hat{\mathbf{k}}\right)\sin\theta\, \dd\theta\, \dd\phi
\]

The components $\hat{\mathbf{i}}$ and $\hat{\mathbf{j}}$ cancel due to the 
$\phi$ factor.

Using the substitution $\mu = \cos\theta$ and denoting $s=\tfrac{D}{R}$:
\[
L(\mu) = \frac{2 R \sqrt{B^2(\mu) - A(\mu) C(\mu)}}{A(\mu)},
\quad
\begin{cases}
A(\mu)=(1-\beta^2)(1-\mu^2)+(\mu+\beta^2\mu+2\beta)^2\\[2pt]
B(\mu)=s\,(1+\beta\mu)(\mu+\beta^2\mu+2\beta)\\[2pt]
C(\mu)=(1+\beta\mu)^2(s^2-1)
\end{cases}
\]

We determine the range of $\mu$ by ensuring the discriminant is positive. 
This results in a degree-4 polynomial in $\mu$. Let $\mu_0$ be the first 
real root greater than $-1$. 

The gravitational field simplifies to:
\[
\gray(\pv, 0) = -2 \pi \hat{d}\,G \rho_{\scriptscriptstyle 0} \int_{-1}^{\mu_0} L(\mu)\, \mu \, \dd\mu
\]

Numerical integrations indicate qualitative agreement with the angular scaling 
of the linearized Liénard--Wiechert field:
\[
\gray(\pv)\ \propto\ -\frac{GM}{D^2}\,\frac{1-\beta^2}{\big(1+\beta\cos\theta\big)^3},
\qquad \beta=v/c
\]

when interpreted as a shape/angle guide rather than an identity. 
Asymptotically (small $\beta$ and/or far field $D\gg R$), the mixed-frames result
approaches the Liénard--Wiechert formula.

\end{example}
\vspace{1em}

\begin{example}[single-frame]
We resolve the same problem as in the previous example using the alternative 
definition of the gravitational field \eqref{eq:rayfield-rel-alt}.

In the $S'$ frame, we parametrize the rays and the sphere to determine the 
amount of mass traversed by each ray. We seek the intersection points 
between the sphere and the ray.
\[
\left.
\begin{array}{rcl}
\text{Ray:} & \xv &= \pv+t\,c\,\uh' \\[3pt]
\text{Sphere:} & x^2+y^2+(z - D + v\, t)^2 &= R^2
\end{array}
\right\},
\quad t \le 0
\]

We change coordinates using $\ell = -t\,c$ (arc length):
\[
\left.
\begin{array}{rcl}
\text{Ray:} & \xv &= \pv-\ell\,\uh' \\[3pt]
\text{Sphere:} & x^2+y^2+(z-D-\beta\, \ell)^2 &= R^2
\end{array}
\right\},
\quad \ell \ge 0
\]

Aberration formulas apply for $\vv$ aligned to the polar axis, but this 
is not the current case, $\vv=-v\,\hat{z}$. However, we can handle this 
case by replacing $\beta\to-\beta$ everywhere in the alternative definition 
\eqref{eq:rayfield-rel-alt}.

We write $\uh'$ in polar coordinates,
\[
\uh' = \big(\sin\theta\cos\phi,\ \sin\theta\sin\phi,\ \cos\theta\big),
\quad \theta \in [0,\pi], \ \phi \in [0,2\pi)
\]

We expand $\xv$ in these coordinates as,
\[
\xv(\uh', \ell) = \Big(-\ell\,\sin\theta\cos\phi,\, -\ell\,\sin\theta\sin\phi,\,-\ell\,\cos\theta\Big)
\]

Finally, we substitute $\xv(\uh',\ell)$ into the sphere equation. 
This yields in a quadratic polynomial in $\ell$
\[
\ell^2 \sin^2\theta+\left(-\cos\theta\,\ell-\beta\ell-D\right)^2=R^2
\]

Rearranging terms by the coefficient of $\ell$, we obtain:
\[
A(\theta)\,\ell^2+2B(\theta)\,\ell+C(\theta)=0,
\quad
\begin{cases}
A(\theta)=1 + 2\beta\cos\theta + \beta^2\\[2pt]
B(\theta)=D\,(\cos\theta + \beta)\\[2pt]
C(\theta)=D^2-R^2
\end{cases}
\]

The chord length traversed by the ray is:
\[
L(\theta) = 2\sqrt{\left(\frac{B(\theta)}{A(\theta)}\right)^2 - \frac{C(\theta)}{A(\theta)}}
\]

Using the substitution $\mu = \cos\theta$ and defining $s=\tfrac{D}{R}$:
\[
L(\mu) =
\frac{2R\,\sqrt{A(\mu)-s^2\big(1-\mu^2\big)}}{A(\mu)},
\quad
A(\mu)=1 + 2\beta\mu + \beta^2
\]

We determine the range of $\mu$ by ensuring the discriminant is positive and 
selecting the first root greater than $-1$:
\[
\mu_0 = \frac{-\beta - \sqrt{(s^2-1)(s^2-\beta^2)}}{s^2}
\]

Using the alternative definition of the gravitational field:
\[
\gray(\pv,t_0)
= -\frac{G}{1-\beta^2}\,\hat{d}\,2\pi\,\rho_{\scriptscriptstyle 0} \int_{-1}^{\mu_0}\underbrace{L(\mu)}_{\text{line integral}}\underbrace{(\mu-\beta)}_{h^{-1}(\uh')}\,\underbrace{(1-\beta\mu)\; \dd\mu}_{\frac{d\Omega}{d\Omega'}d\Omega'}
\]

The numerical results obtained are the same as those in Example \ref{example:mixed_frames}.

\end{example}

\section*{Final Notes}

We outlined a ray-transport construction that reproduces the Newtonian field in 
the static case and there are indications that it can be extended to the 
relativistic case. The following are aspects to consider in future research. 
These pointers are deliberately high-level and speculative, offered as indications 
of where the framework might be extended rather than as claims or detailed 
developments.

\textbf{Inner case}. The equivalence result in Proposition \ref{prop:equivalence}
is valid only outside the support of the mass density. Verify that the same
result is obtained inside the support.

\textbf{Cross-links to other areas of mathematics}. The exploration of links 
with other mathematical frameworks remains open. In particular, it is natural 
to relate our line-integral operator to integral geometry and to the Radon/X-ray 
transform.

\textbf{$n$-dimensional manifold instead of a ray}. Consider generalizing the
ray concept to an $n$-dimensional manifold. This generalization could be 
particularly useful in non-Euclidean geometries. For instance, in the 
one-dimensional case, the manifold could correspond to a geodesic. Such an 
approach would facilitate connections with geometric formulations, such as 
those found in General Relativity.

\textbf{Equivalence with the standard model}. The equivalence between the 
ray-based model and the standard formulation of gravity (GR) should be 
demonstrated, or at least the domain of validity of the ray-based model should 
be delimited. This could involve expanding the notion of a ray or 
reformulating or extending the definition of the ray-based field.

\begin{thebibliography}{9}
\bibitem{Folland}
G.~B. Folland, \emph{Real Analysis: Modern Techniques and Their Applications}, 2nd ed., Wiley, 1999.

\bibitem{Rudin}
W.~Rudin, \emph{Real and Complex Analysis}, 3rd ed., McGraw--Hill, 1987.

\bibitem{LandauLifshitz1}
L.~D. Landau and E.~M. Lifshitz, \emph{Mechanics}, 3rd ed., Pergamon, 1976.

\bibitem{LandauLifshitz2}
L.~D. Landau and E.~M. Lifshitz, \emph{The Classical Theory of Fields}, 4th ed., Pergamon, 1975.
\end{thebibliography}

\end{document}
